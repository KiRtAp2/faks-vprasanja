\section{Mestni zapis števil}

\vprasanje{Kako zapišemo število $n \in \N$ z osnovo $c$?}

Število zaporedoma delimo s $c$, ostanek pri deljenju pa zapišemo na ustrezno mesto. Ker potence $c, c^2, c^3, \ldots$ naraščajo preko vsake meje, lahko tako zapišemo vsako naravno število.

\vprasanje{Kako razširimo mestni zapis na racionalna števila? Kakšna slabost se tako prikaže?}

Tako, da primerjamo število tudi z negativnimi eksponenti. Slabost je, da tako dobljen zapis ni enoličen.

\section{Zaporedja}

\vprasanje{Kaj je aritmetično zaporedje? Kako izračunamo vsoto prvih $n$ členov takega zaporedja?}

To je zaporedje, kjer je razlika med sosednjima členoma konstantna.

\[
	S_n = \frac{n(a_0 + a_n)}{2} = \frac{n(2a + (n-1)d)}{2} = na + \frac{n(n-1)}{2}d
\]

\vprasanje{Kaj je geometrijsko zaporedje? Kako izračunamo vsoto prvih $n$ členov takega zaporedja?}

To je zaporedje, kjer je kvocient med dvema sosednjima členoma konstanten.

\[
	S_n = \frac{a_1 (q^n - 1)}{q-1}
\]

Če pa je $q=1$:

\[
	S_n = na_1
\]

\vprasanje{Kako izračunamo vsoto neskončnega geometrijskega zaporedja?}

Če je $\left|q\right| < 1$:
\[
	S = \frac{a_1}{1-q}
\]

\section{Razstavljanje izrazov}

\vprasanje{Razstavi $a^2-b^2$, $a^3-b^3$, $a^3+b^3$, $a^n-b^n$ in $a^n+b^n$ za nek $n \in \N$. Ali obstaja kakšna omejitev glede izbire $n$?}

\[
	a^2-b^2 = (a-b)(a+b)
\]

\[
	a^3-b^3 = (a-b)(a^2 + ab + b^2)
\]

\[
	a^3 + b^3 = (a+b)(a^2-ab+b^2)
\]

\[
	a^n - b^n = (a-b)(a^{n-1} + a^{n-2}b + \ldots + b^{n-1})
\]

\[
	a^n + b^n = (a+b)(a^{n-1} - a^{n-2}b + \ldots + b^{n-1})
\]

Zadnji razcep velja le za lihe $n$.

\section{Kompleksna števila}

\vprasanje{Kaj je kompleksno število? Kaj je njegovo konjugirano število?}

Kompleksno število je urejen par realnih števil, zapišemo ga z $z = x+yi;\, x,y \in \R, i^2 = -1$.

Njegovo konjugirano število je $\overline{z} = x - yi$.
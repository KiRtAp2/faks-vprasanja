\section{Vektorski prostor $\R^3$}

\subsection{Koordinatni sistem in vektorji v prostoru}

\vprasanje{Kakšen je pozitivno orientiran koordinatni sistem?}

Tak, kjer je ordinatna os za $90\degree$ pozitivno rotirana od abscisne osi.

\vprasanje{Kaj sestavlja koordinatni sistem v prostoru?}

Tri medsebojno paroma pravokotne številske premice, ki se sekajo v koordinatnem izhodišču.

\vprasanje{Kaj je krajevni vektor točke $T = (x,y,z) \in \R^3$?}

To je usmerjena daljica z začetkom v izhodišču in koncem v točki $T$.

\vprasanje{Kaj je vektor?}

Vektor $\vec{a} = (x, y, z)$ je množica vseh usmerjenih daljic, ki jih dobimo z vzporednim premikom krajevnega vektorja do točke $T(x, y, z)$.

\vprasanje{Kaj je linearna kombinacija vektorjev $\vec{a}_1, \vec{a}_2, \ldots, \vec{a}_n$?}

To je vsak izraz oblike $\alpha_1 \vec{a}_1 + \alpha_2 \vec{a}_2 + \ldots + \alpha_n \vec{a}_n$, kjer so $\alpha_1, \alpha_2, \ldots, \alpha_n \in \R$.

\vprasanje{Kdaj so vektorji $\vec{a}_1, \vec{a}_2, \ldots, \vec{a}_n$ linearno neodvisni? Kdaj sta dva vektorja $\vec{a}$ in $\vec{b}$ linearno odvisna?}

Kadar nobeden izmed vektorjev ni enak kakšni linearni kombinaciji ostalih.

Dva vektorja sta linearno odvisna, kadar velja $\vec{a} = \alpha \vec{b}$ ali $\vec{b} = \beta \vec{a}$ za neka $\alpha, \beta \in \R$. To je natanko takrat, ko sta vzporedna.

\vprasanje{Kaj je baza prostora $\R^3$? Kaj je \textit{standardna baza}?}

Baza prostora je množica treh linearno neodvisnih vektorjev.

Standardna baza so vektorji $\vec{i} = (1, 0, 0), \vec{j} = (0, 1, 0), \vec{k} = (0, 0, 1)$.

\subsection{Skalarni produkt}

\vprasanje{Kaj je skalarni produkt vektorjev $\vec{a}_1 = (x_1, y_1, z_1)$ in $\vec{a}_2 = (x_2, y_2, z_2)$?}

To je število $\vec{a}_1 \cdot \vec{a}_2 = x_1 x_2 + y_1 y_2 + z_1 z_2$.

\vprasanje{Kaj je \textit{norma} vektorja $\vec{a}$? Kaj predstavlja?}

To je število $\| \vec{a} \| = \sqrt{\vec{a} \cdot \vec{a}}$. Predstavlja dolžino vsake usmerjene daljice, ki predstavlja vektor $\vec{a}$.

\vprasanje{Naštej 4 lastnosti skalarnega produkta.}

\begin{enumerate}
	\item Komutativnost: $\vec{a} \cdot \vec{b} = \vec{b} \cdot \vec{a}$.
	\item Distributivnost: $(\vec{a} + \vec{b}) \cdot \vec{c} = \vec{a} \cdot \vec{c} + \vec{b} \cdot \vec{c}$.
	\item Homogenost: $(\alpha \vec{a}) \cdot \vec{b} = \alpha (\vec{a} \cdot \vec{b})$.
	\item Pozitivna definitnost: $\vec{a} \cdot \vec{a} \ge 0$ in $\vec{a} \cdot \vec{a} = 0 \nt \vec{a} = \vec{0}$.~
\end{enumerate}

\vprasanje{Kako izračunaš kot med dvema vektorjema $\vec{a}$ in $\vec{b}$? Kaj je kriterij za pravokotnost vektorjev?}

\[
	\cos \phi = \frac{\vec{a} \cdot \vec{b}}{\norm{\vec{a}} \norm{\vec{b}}}
\]
Kjer je $\phi$ kot med vektorjema.

Kriterij za pravokotnost: $\vec{a} \perp \vec{b} \nt \vec{a} \cdot \vec{b} = 0$.

\subsection{Vektorski produkt}

\vprasanje{Kaj je vektorski produkt vektorjev $\vec{a}$ in $\vec{b}$?}

To je vektor $\vec{a} \x \vec{b}$, za katerega velja:
\begin{itemize}
	\item pravokoten je na $\vec{a}$ in na $\vec{b}$.
	\item dolžina je ploščina paralelograma, napetega na krajevna vektorja $\vec{a}$ in $\vec{b}$.
	\item trojica $(\vec{a}, \vec{b}, \vec{a} \x \vec{b})$ je pozitivno orientirana.
\end{itemize}

\vprasanje{Kakšen je predpis za vektorski produkt?}

\[
	(x_1, y_1, z_1) \x (x_2, y_2, z_2) = 
	\begin{vmatrix}
		\vec{i} & \vec{j} & \vec{k} \\
		x_1 & y_1 & z_1 \\
		x_2 & y_2 & z_2
	\end{vmatrix}
\]

\vprasanje{Povej 3 lastnosti vektorskega produkta.}

\begin{itemize}
	\item Antikomutativnost: $\vec{a} \x \vec{b} = - \vec{b} \x \vec{a}$.
	\item Distributivnost: $\vec{a} \x (\vec{a} + \vec{c}) = \vec{a} \x \vec{b} + \vec{a} \x \vec{c}$.
	\item Homogenost: $(\alpha \vec{a}) \x \vec{b} = \vec{a} \x (\alpha \vec{b}) = \alpha (\vec{a} \x \vec{b})$
\end{itemize}

\subsection{Mešani produkt}

\vprasanje{Kaj je mešani produkt vektorjev $\vabc$ ? Kakšen je njegov predpis?}

To je število $[ \vabc ] = (\vec{a} \x \vec{b}) \cdot \vec{c}$.

\[
	[\vec{a}, \vec{b}, \vec{c}] = \begin{vmatrix}
		x_1 &  y_1 & z_1 \\
		x_2 & y_2 & z_2 \\
		x_3 & y_3 & z_3
	\end{vmatrix}
\]

\vprasanje{Kakšna je geometrijska interpretacija mešanega produkta?}

Mešani produkt $[\vabc]$ je volumen paralelepipeda, napetega na vektorje $\vec{a}$, $\vec{b}$ in $\vec{c}$, pomnožen z orientacijo urejene trojice $(\vabc)$.

\vprasanje{Kako izračunamo prostornino nepravilnega tetraedra, določenega z vektorji $\vabc$ ?}

\[
	V = {{1 \over 6} \left| [\vabc] \right|}
\]

\vprasanje{Povej 2 lastnosti mešanega produkta.}

\begin{enumerate}
	\item Asociativnost v vseh faktorjih: $[\vec{a}_1 + \vec{a}_2, \vec{b}, \vec{c}] = [\vec{a}_1, \vec{b}, \vec{c}] + [\vec{a}_2, \vec{b}, \vec{c}]$.
	\item Homogenost: $\alpha [\vabc] = [\alpha \vec{a}, \vec{b}, \vec{c}] = [\vec{a}, \alpha \vec{b}, \vec{c}] = [\vec{a}, \vec{b}, \alpha \vec{c}]$.
\end{enumerate}

\vprasanje{Kaj je Lagrangeva identiteta?}

\[
	(\vec{a} \x \vec{b}) \cdot (\vec{c} \x \vec{d}) = \begin{vmatrix}
		\vec{a} \vec{c} & \vec{a} \vec{d} \\
		\vec{b} \vec{c} & \vec{b} \vec{d}
	\end{vmatrix}
\]

\section{Premice in ravnine v $\R^3$}

\subsection{Enačbe ravnin}

\vprasanje{Kaj je enačba ravnine?}

Enačba ravnine $\Sigma$ je taka enačba v spremenljivkah $x,y,z$, da velja:
\begin{itemize}
	\item Če točka $T(a,b,c)$ leži na $\Sigma$, potem $a, b, c$ zadoščajo enačbi.
	\item Če $T \notin \Sigma$, pa $a, b, c$ ne zadoščajo enačbi.
\end{itemize}

\vprasanje{Kaj je normala ravnine? Koliko normal ima ravnina? Kaj še potrebujemo, da natanko določimo ravnino, poleg normale?}

Normala ravnine je poljuben neničelen vektor, ki je pravokoten na ravnino.

Ravnina ima več normal, vse izmed katerih so si vzporedne.

Ravnina je natanko določena z normalo in eno točko na ravnini.

\vprasanje{Kakšne splošne oblike je enačba ravnine? Kako iz te oblike preberemo normalo? Ali je enačba ravnine enolična?}

Splošna oblika: $ax + by + cz + d = 0$, $a, b, c$ niso vsi $0$.

Normala take ravnine je $(a, b, c)$.

Enačba ravnine ni enolična, saj lahko enačbo pomnožimo z poljubnih neničelnim skalarjem, in še vedno predstavlja isto ravnino.

\vprasanje{Kaj je \textit{normalna enačba ravnine}? Kako je z njeno enoličnostjo?}

To je poseben primer enačbe ravnine $ax + by + cz + d = 0$, kjer ima normala $(a, b, c)$ dolžino $1$. Je enolična do predznaka natančno, lahko jo pomnožimo z $-1$.

\vprasanje{Podaj enačbo ravnine skozi tri nekolinearne točke $(x_0, y_0, z_0), (x_1, y_1, z_1), (x_2, y_2, z_2)$.}

\[
	\begin{vmatrix}
		x-x_0 & y - y_0 & z - z_0 \\
		x_1-x_0 & y_1 - y_0 & z_1 - z_0 \\
		x_2-x_0 & y_2 - y_0 & z_2 - z_0
	\end{vmatrix} = 0
\]

\subsection{Razdalja do ravnine}

\vprasanje{Kaj je razdalja med točko $T_1$ in ravnino? Navedi njeno formulo z uporabo normale $\vec{n}$, krajevnega vektorja do točke na ravnini $\vec{r}_0$ ter krajevnega vektorja do točke $T_1$, $\vec{r}_1$.}

To je najkrajša razdalja $\Delta$ med $T_1$ in kakšno točko na ravnini.

\[
	\Delta = \left| \frac{\vec{n}(\vec{r}_1 - \vec{r}_0)}{\norm{\vec{n}}} \right|
\]

\vprasanje{Navedi formulo za razdaljo točke $T(x_0,y_0,z_0)$ od ravnine z enačbo $ax + by + cz +d = 0$.}

\[
	\Delta = \left| \frac{ax_0 + by_0 + cz_0 + d}{\sqrt{a^2 + b^2 + c^2}} \right|
\]

\vprasanje{Kaj je razdalja med dvema vzporednima ravninama?}

To je razdalja med poljubno točko na eni ravnini in drugo ravnino.

\vprasanje{Kaj je razdalja med premico in njej vzporedno ravnino?}

To je razdalja med poljubno točko na premici in ravnino.

\subsection{Enačbe premic}

\vprasanje{Kaj je enačba premice v prostoru?}

Sistem dveh linearnih enačb v spremenljivkah $x, y, z$, da velja: $T(a,b,c)$ leži na premici natanko takrat, ko $a, b, c$ zadostujejo obema enačbama.

\vprasanje{Kaj je smerni vektor premice? Kaj še potrebujemo, da premico popolnoma definiramo?}

Smerni vektor premice je poljuben neničelni vektor, ki je premici vzporeden. Da enolično določimo premico, potrebujemo še eno točko na premici.

\vprasanje{Povej vektorsko enačbo premice ter enačbo premice po komponentah.}

\[
	\vec{r} = \vec{r}_0 + \lambda \vec{s}
\]

Kjer je $\vec{r}$ poljuben vektor, $\vec{r}_0$ krajevni vektor do točke na premici, $\lambda$ neko realno število in $\vec{s}$ smerni vektor premice.

\[
	\frac{x-x_0}{a} = \frac{y-y_0}{b} = \frac{z-z_0}{c}
\]

Kjer so $x, y, z$ koordinate do poljubne točke, $x_0, y_0, z_0$ koordinate točke na premici ter $a, b, c$ komponente smernega vektorja premice.

\subsection{Razdalja do premice}

\vprasanje{Povej enačbo za razdaljo točke $T_1$ s krajevnim vektorjem $\vec{r}_1$ od premice z enačbo $\vec{r} = \vec{r}_0 + \lambda \vec{s}$.}

\[
	\Delta = \frac{\norm{(\vec{r}_1 - \vec{r}_0) \x \vec{s}}}{\norm{\vec{s}}}
\]

\vprasanje{Čemu je enaka razdalja med dvema vzporednima premicama?}

Razdalji med poljubno točko na eni premici in drugo premico.

\vprasanje{Povej enačbo za razdaljo med dvema mimobežnima premicama. Kdaj se nevzporedni premici sekata?}

\[
	\Delta = \frac{\left|[\vec{r}_2 - \vec{r_1}, \vec{s}_1, \vec{s}_2]\right|}{\norm{\vec{s}_1 \x \vec{s}_2}}
\]

Kjer sta $\vec{r}_1, \vec{r}_2$ krajevna vektorja do dveh točk na premicah (ena točka na premico), $\vec{s}_1$ in $\vec{s}_2$ pa smerna vektorja premic.

Dve nevzporedni premici se sekata, kadar je navedeni mešani produkt $0$.

\section{Osnovne algebraične strukture}

\subsection{Operacije na množicah}

\vprasanje{Kaj je binarna notranja operacija na množici $A$? Kako v splošnem imenujemo njen izhod?}

To je preslikava $A \x A \desno A, (x, y) \slika x \circ y$. $x\circ y$ v splošnem imenujemo \textit{kompozitum}.

\vprasanje{Kaj je dvočlena zunanja operacija?}

To je preslikava množic $A$ in $R$ s predpisom $R\x A \desno A.$

\vprasanje{Kako imenujemo množico, na kateri je definirana vsaj ena operacija?}

Algebraična struktura.

\vprasanje{Kdaj je $a \in A$ obrnljiv?}

Kadar obstaja inverz $a^{-1} \in A$, da je $a^{-1} \circ a = a \circ a^{-1} = e$.

\subsection{Grupe}

\vprasanje{Kaj so grupoid, polgrupa in monoid? Za vsakega povej primer.}

Grupoid je neprazna množica z binarno operacijo $(A, \circ)$, npr.~$(\Z, -)$.

Polgrupa je grupoid, kjer je operacija asociativna, npr.~$(\N, +)$.

Monoid je polgrupa z enoto, npr.~$(\R, \cdot)$.

\vprasanje{Kaj je inverz kompozituma $a\circ b$ obrnljivih elementov $a, b \in A$ v monoidu $(A,\circ)$?}

\[
	b^{-1} \circ a^{-1}
\]

\vprasanje{Kaj je grupa? Podaj primer. Kdaj je grupa komutativna?}

To je monoid, v katerem je vsak element obrnljiv, npr. $(\Z, +)$. Grupa je komutativna, kadar velja $a \circ b = b \circ a$ za vsaka $a, b$ elementa grupe.

\vprasanje{Do katerega $n$ so končne grupe velikosti $n$ komutativne?}

Do $n=5$.

\vprasanje{Kaj je permutacija končne množice $A$? Kako označimo grupo permutacij množice $A$ z $n$ elementi in operacijo $\circ$? Koliko elementov ima?}

Permutacija na množici $A$ je bijektivna preslikava $A \desno A$. Množico vseh permutacij označimo z $S(A)$, grupo $(S(A), \circ)$ pa z $S_n$. Imenujemo jo \textit{simetrična grupa} reda $n$, ima pa $n!$ elementov.

\vprasanje{V kakšni obliki običajno pišemo permutacije?}

\[
	\Pi = \begin{pmatrix}
		1 & 2 & 3 & \ldots & n \\
		\Pi(1) & \Pi(2) & \Pi(3) & \ldots & \Pi(n)
	\end{pmatrix}
\]

\vprasanje{Katere grupe permutacij s kompozitumom so komutativne?}

Samo $S_2$.

\vprasanje{Kaj je cikel? Kako ga označimo?}

Naj bodo $a_1, a_2, \ldots, a_k \in \{1, 2, \ldots, n\}$ paroma različni. Permutacija $\sigma \in S_n$, definirana s predpisom $\sigma(a_1) = a_2, \sigma(a_2) = a_3, \ldots, \sigma(a_{k-1}) = a_k, \sigma(a_k) = a_1$ in $\sigma(x) = x$ za $x \in \{1, 2, \ldots, n\} \brez {a_1, a_2, \ldots, a_k}$ se imenuje cikel dolžine $k$. Označimo ga z $\begin{pmatrix}
a_1 & a_2 & \ldots & a_k
\end{pmatrix}$.

\vprasanje{Kdaj sta dva cikla disjunktna? Kakšno lastnost imata?}

Cikla sta disjunktna, če pripadajoči množici elementov, ki v ciklu niso slike samih sebe, nimata skupnih elementov. Disjunktna cikla vedno komutirata.

\vprasanje{Kako se imenuje cikel z dolžino $2$?}

Transpozicija.

\vprasanje{Kako je definiran znak permutacije?}

$s(\id{{}} ) = 1$.

Za cikel $\sigma$ dolžine $k$ je $s(\sigma) = (-1)^{k+1}.$

Za permutacijo $\Pi$, razdeljeno na cikle, je $s(\Pi) = s(\sigma_1) s(\sigma_2) \ldots s(\sigma_m)$.

\vprasanje{Kakšen je znak produkta permutacije s transpozicijo? Kdaj je permutacija soda in kdaj liha? Kakšen znak ima inverz sode permutacije?}

Nasproten znaku permutacije. Permutacija je soda, če jo lahko zapišemo kot produkt sodega števila transpozicij. Permutacija je liha, če jo lahko zapišemo kot produkt lihega števila transpozicij. Inverz sode permutacije ima enak znak kot soda permutacija.

\vprasanje{Kaj je \textit{alternirajoča grupa} reda $n$? Kakšne moči je?}

To je grupa \[
	A_n = \{\Pi \in S_n | \, \Pi \, \text{je soda permutacija}\}
\] z močjo $n! \over 2$.

\subsection{Podgrupe}

\vprasanje{Kaj je podgrupa?}

Naj bo $(G, \cdot)$ grupa. Neprazna podmnožica $H \subseteq G$ je podgrupa $G$, kadar velja:

\begin{itemize}
	\item Zaprtost za množenje: Če sta $a, b \in H$, potem $ab \in H$.
	\item Zaprtost za invertiranje: Če je $a \in H$, je tudi $a^{-1} \in H$.
\end{itemize}

\vprasanje{Kateri dve podgrupi ima vsaka grupa $G$ moči $2$ ali več?}

\textit{Neprava} podgrupa: $G$.

\textit{Trivialna} podgrupa: $\{\text{enota}\}$.

\vprasanje{Kdaj je neprazna podmnožica $H$ grupe $(G, \cdot)$ tudi podgrupa $G$?}

Kadar za vsaka $a, b \in H$ tudi $ab^{-1} \in H$.

\subsection{Homomorfizem grup}

\vprasanje{Kaj je homomorfizem grup $(G, \circ)$ in $(H, *)$? Povej primer.}

To je preslikava $f: G \desno H$, za katero za vsaka $a, b \in G$ velja $f(q\circ b) = f(a) * f(b)$; npr.~preslikava $f: (\Z, +) \desno (\Q \brez \{0\}, \cdot), f(x) = 2^x$.

\vprasanje{Kako homomorfizem grup preslika enoto in inverz?}

Enoto preslika v enoto, inverz pa v inverz.

\vprasanje{Definiraj izraze: \textit{monomorfizem}, \textit{epimorfizem}, \textit{izomorfizem}, \textit{endomorfizem} in \textit{avtomorfizem}.}

Monomorfizem je injektiven homomorfizem.

Epimorfizen je surjektiven homomorfizem.

Izomorfizem je bijektiven homomorfizem.

Endomorfizem je homomorfizem $G \desno G$.

Avtomorfizem je bijektivni endomorfizem.

\vprasanje{Povej primer notranjega avtomorfizma grupe $G$.}

\[
	f_a : G \desno G
\]

\[
	x \slika a x a^{-1}
\]

\vprasanje{Kdaj sta grupi $G$ in $H$ izomorfni? Kako to označimo?}

Kadar obstaja izomorfizem $G \desno H$. To označimo $G \cong H$.

\vprasanje{Kaj je slika in kaj jedro homomorfizma $f$?}

Slika homomorfizma je \[
	\im f = \{f(x); x \in G\}
\]

Jedro homomorfizma je \[
	\ker f = \{x \in G; f(x) = 1\}
\]

\vprasanje{Kako preveriš injektivnost homomorfizma $f$?}

$f$ je injektiven natanko tedaj, ko je $\ker f = \{1\}$.

\section{Kolobarji}

\vprasanje{Definiraj kolobar.}

Neprazna množica $K$ z operacijama $+, \cdot$ se imenuje \textit{kolobar}, kadar velja:
\begin{itemize}
	\item $(K, +)$ je Abelova grupa
	\item $(K, \cdots)$ je polgrupa
	\item Veljata distributivnostna zakona: $a(b+c) = ab+ac$ in $(b+c)a = ba + ca$.
\end{itemize}

\vprasanje{Kaj je \textit{kolobar z enoto} in kaj \textit{komutativen kolobar}?}

Kolobar z enoto je kolobar z enoto za množenje.

Komutativen kolobar je takšen, kjer je množenje komutativno.

\vprasanje{Kaj je \textit{delitelj niča}? Povej primer.}

Naj bo $(K, +, \cdot)$ kolobar in $a, b \in K$ neničelna elementa, za katera velja $ab = 0$. $a$ je levi delitelj niča, $b$ pa desni delitelj niča.

Primer: $2, 3$ sta delitelja niča v $\Z_6$.

\vprasanje{Kaj je \textit{obseg}? Kaj je \textit{polje}? Koliko največ deliteljev niča ima obseg?}

Kolobar $(K, +, \cdot)$ je obseg, kadar ima enoto $1 \ne 0$ in za vsak $a \in K \brez \{0\}$ obstaja $a^{-1} \in K$, da je $aa^{-1} = a^{-1}a = 1$.

Polje je komutativen obseg.

V obsegu ni deliteljev niča.

\vprasanje{Kdaj je $\Z_n$ polje?}

Kadar je $n$ praštevilo.

\vprasanje{Definiraj podkolobar.}

$H \subseteq K$ je podkolobar kolobarja $(K, +, \cdot)$, kadar je $(H,+)$ podgrupa Abelove grupe $(K, +)$ in je $H$ zaprta za množenje.

\vprasanje{Definiraj homomorfizem kolobarjev.}

Naj bosta $K$ in $L$ kolobarja. Preslikava $f: K \desno L$ je homomorfizem kolobarjev, kadar velja $f(a+b) = f(a)+f(b)$ in $f(ab) = f(a)f(b)$ za vsaka $a, b \in K$.

\vprasanje{Definiraj podobseg in podpolje.}

Naj bo $O$ obseg. Neprazna podmnožica $O' \subseteq O$ je podobseg obsega $O$, kadar je podkolobar in za vsak $a \in O' \brez \{0\}$ velja $a^{-1} \in O'$. Če je $O$ komutativen, govorimo o podpolju.

\section{Vektorski prostori}

\vprasanje{Definiraj vektorski prostor nad poljem $F$.}

To je Abelova grupa $(V,+)$ skupaj z zunanjo operacijo $F\x V \desno V$, za katero velja:
\begin{itemize}
	\item distributivnost v skalarju
	\item distributivnost v vektorju
	\item homogenost
	\item $\forall v \in V. 1v = v$
\end{itemize}

\vprasanje{Definiraj vektorski podprostor prostora $V$ nad poljem $F$.}

Neprazna podmnožica $W \subseteq V$ je vektorski podprostor prostora $V$, kadar velja:
\begin{itemize}
	\item Zaprtost za seštevanje
	\item Zaprtost za množenje s skalarjem
\end{itemize}

\vprasanje{Kaj je trivialni vektorski podprostor?}

$\{0\}$

\vprasanje{Kaj je linearna kombinacija?}

To je vsak zapis oblike $\alpha_1 x_1 + \alpha_2 x_2 + \ldots + \alpha_n x_n$, kjer so $x_1, x_2, \ldots, x_n$ vektorji in $\alpha_1, \alpha_2, \ldots, \alpha_n$ skalarji.
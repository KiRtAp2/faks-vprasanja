\section{Števila}

\subsection{Naravna števila}

\vprasanje{Kako označimo naslednjika naravnega števila $n$?}

\[n^+\]

\vprasanje{Kaj so Peanovi aksiomi?}

So aksiomi, ki definirajo množico $\N$ skupaj s pravilom, ki vsakemu naravnemu številu $n$ priredi naslednjika $n^+$ ($n, n^+ \in \N$):
\begin{itemize}
	\item Za vsaka $n, m \in \N$ in $m^+ = n^+$ velja $m=n$
	\item Obstaja število $1 \in \N$, ki ni naslednjik od nobenega naravnega števila
	\item \textbf{Aksiom popolne indukcije}: Če $A \subset \N$ in če je $1 \in A$ in če je za vsak $n \in A$ tudi $n^+ \in A$, potem je $A = \N$.
\end{itemize}

\vprasanje{Kdaj je množica \textit{dobro urejena}? Povej primer takšne množice. Povej primer množice, ki ni dobro urejena.}

Kadar ima vsaka neprazna podmnožica najmanjši element. Dobro urejena je npr. $\N$, ne pa $\Z$.

\subsection{Racionalna števila}

\vprasanje{Kdaj ulomka $\frac{m}{n}$ in $\frac{k}{l}$ predstavljata isto število?}

Kadar je $ml = nk$.

\vprasanje{Kaj so ulomki?}

Množica $\Z \x \N = \{(m, n); m \in \Z, n \in \N \}$.

\vprasanje{Kaj je racionalno število?}

Množico $\Z \x \N$ razdelimo na ekvivalenčne razrede:
\[
	(m, n) \sim (k, l) \nt ml = nk \quad \forall m, k \in \Z, \forall n, l \in \N
\]

Racionalno število je ekvivalenčni razred urejenih parov in ga označimo z $\frac{m}{n}$:

\[
	\frac{m}{n} = \{(k, l), ml = nk\}
\]

\vprasanje{Kateri trije aksiomi veljajo za grupe? Kateri dodatni velja za Abelove grupe?}

Za grupe veljajo naslednji aksiomi: (prikazani simboli za množico $A$ in dvočlen operator $+: A \x A \rightarrow A$)
\begin{enumerate}
	\item Asociativnost: $(a+b)+c = a+(b+c) \quad \forall a, b, c \in A$
	\item Obstoj enote: obstaja $0 \in A$, tako da za vsak $a \in A$ velja $0+a = a+0 = a$
	\item Obstoj inverznega elementa: za vsak $a \in A$ obstaja inverzni element $-a \in A$, tako da velja $a + (-1) = (-a) + a = 0$
\end{enumerate}

Za \textit{Abelove} (oz. \textit{komutativne}) grupe velja tudi aksiom komutativnosti:

\[
	a + b = b + a \quad \forall a, b \in A
\]

\vprasanje{Povej primer Abelove grupe.}

Abelova grupa: $(\Q, +)$ ali $(\Q \brez \{0\}, \cdot)$

\vprasanje{Kako označimo enoto in inverzni element $a$ za seštevanje in množenje?}

Za seštevanje: $0$, $-a$

Za množenje: $1$, $a^{-1}$

\vprasanje{Povej pravilo krajšanja za seštevanje v grupi $A$ z operacijo $+: A \x A \rightarrow A$}

Naj bodo $a, x, y \in A$. Če velja $a + x = a + y$, potem je $x=y$.

\vprasanje{Kaj je \textit{komutativen obseg}? Kako ga še drugače imenujemo? Povej primer.}

To je množica $A$ z operacijama $+, \cdot$, kjer je $(A, +)$ Abelova grupa za seštevanje, $(A \brez \{0\}, \cdot)$ Abelova grupa za množenje, veljata pa še dva aksioma:

\begin{enumerate}
	\item $1 \ne 0$ (enota za seštevanje ni enaka enoti za množenje)
	\item Aksiom distributivnosti: Za vse $a, b, c \in A$ velja $a(b+c) = ab+ac$.
\end{enumerate}

Komutativen obseg imenujemo tudi \textit{polje}. Primer je $(\Q, +, \cdot)$~

\vprasanje{Kaj je \textit{urejen obseg}?}

To je obseg, ki ima urejenost, ki ustreza naslednjima aksiomoma:

\begin{enumerate}
	\item Za vsako število $a \in A, a \ne 0$ velja, da je natanko eno od števil $a, -a$ pozitivno.
	\item Za vsaki pozitivni števili $a, b \in A$ sta $a + b$ in $a \cdot b$ pozitivni.
\end{enumerate}

\vprasanje{Kako je definirana urejenost v urejenem obsegu?}

\[
	a < b \nt b - a \; \text{je pozitivno} \quad a, b \in A
\]

\subsection{Realna števila}

\vprasanje{Kaj je Dedekinov rez? Definiraj množico realnih števil. Kako jo označimo?}

To je vsaka podmnožica $A \subset \Q$, za katero velja:
\begin{enumerate}
	\item $A \ne \emptyset, A \ne \Q$
	\item za vsak $p \in A$ in za vsak $q \in \Q, q < p$, je tudi $q \in A$
	\item za vsak $p \in A$ obstaja $q \in A$, da je $q > p$
\end{enumerate}

Realna števila so množica vseh rezov. Označimo jo z $\R$.

\vprasanje{Kako je definiran nasprotni element reza $A$?}

$-A = \{p \in \Q; \text{ obstaja } r \in \Q, r > 0: -p-r \notin A \}$.

\vprasanje{Kdaj je rez $A$ pozitiven?}

Kadar $0^* \subset A \land 0^* \ne A$.

\vprasanje{Naj bo $B$ urejen obseg. Kdaj je množica $A \subset B$ navzgor omejena? Kako imenujemo najmanjšo od vseh zgornjih mej, če obstaja?}

Kadar obstaja $M \in B$, da velja $a \le M$ za vsak $a \in A$.

Najmanjšo zgornjo mejo imenujemo \textit{natančna zgornja meja} ali \textit{supremum} množice $A$; označimo ga s $\sup A$.

\vprasanje{Kaj je maksimum množice $A$?}

To je največji element množice $A$, če obstaja. Označimo ga s $\max A$.

\vprasanje{Kako imenujemo najmanjšo od vseh spodnjih mej množice? Kaj je minimum množice?}

Natančno spodnjo mejo imenujemo \textit{infimum} in jo označimo $\inf A$ (kjer je $A$ množica).

Minimum množice $A$ je najmanjši element množice, če obstaja. Označimo ga z $\min A$.

\vprasanje{Kaj pravi Dedekindov aksiom? Kateri obseg ga izpolnjuje?}

Vsaka neprazna množica navzgor omejena podmnožica v $A$ ima natančno zgornjo mejo.

Izpolnjuje ga obseg $(\R, +, \cdot, <)$.

\vprasanje{Kakšna je razlika med algebrajskimi in transcendentnimi števili?}

Algebrajska števila so rešitve polinomskih enačb s celimi koeficienti.

Transcendentna števila so vsa ostala racionalna števila.

\vprasanje{Katere so posledice Dedekindovega aksioma?}

\begin{enumerate}
	\item $\Z$ v $\R$ ni navzgor omejena.
	\item Za vsako realno število $a$ obstaja $m \in \Z, m > a$.
	\item (\textit{Arhimedska lastnost}): Naj bosta $a, b \in \R, a, b > 0$. Potem obstaja $n \in \N, na > b$.
	\item Naj bo $a$ pozitivno realno število. Potem obstaja $n \in \N, \nicefrac{1}{n} < a$
\end{enumerate}

\vprasanje{Kakšen je zaprti in kakšen odprti interval?}

Zaprti: $[a, b]$

Odprti: $(a, b)$

\vprasanje{Kaj je $\epsilon$-okolica točke $a$? Kaj je okolica točke $a$?}

$\epsilon$-okolica je interval $(a-\epsilon, a+\epsilon)$.

Okolica točke $a$ je vsaka taka podmnožica v $\R$, ki vsebuje kakšno $\epsilon$-okolico točke $a$.

\vprasanje{Kaj je absolutna vrednost realnega števila $x$?}

\[
	\left|x\right| = \begin{cases}
		x, & x \ge 0\\
		-x, & x \le 0
	\end{cases}
\]

\vprasanje{Kaj je trikotniška neenakost?}

\[
	\left|x+y\right| \le \left|x\right| + \left|y\right|
\]

\subsection{Kompleksna števila}

\vprasanje{Kaj je kompleksno število $\alpha$? Kako označimo množico vseh kompleksnih števil?}

To je urejen par realnih števil $(a,b)$. Množico vseh kompleksnih števil označimo z $\C = \R \x \R$.

\vprasanje{Kaj je geometrijski pomen absolutne vrednosti?}

$|z|$ je razdalja točke $z$ od koordinatnega izhodišča.

\vprasanje{Kako je določen polarni zapis kompleksnega števila?}

Z razdaljo od izhodišča ter poltrakom skozi izhodišče, na katerem se število nahaja; ta poltrak je določen s pozitivnim kotom od pozitivnega dela abscisne osi.

\vprasanje{Kaj je argument kompleksnega števila $z$?}

To je kot $\phi$ v polarnem zapisu $z = \cos \phi + i \sin \phi$, kjer je $\phi \in [0, 2\pi)$.

\vprasanje{Katera pogoja morata biti izpolnjena, da je $r(x+yi)$ polarni zapis kompleksnega števila?}

$r>0$ in $x^2 + y^2 = 1$.

\vprasanje{Povej vse rešitve enačbe $w^n = z$ za $z \in \C \brez \{0\}, n \in \N$.}

\[
	w_k = \sqrt[n]{r} \,(\cos (\frac{\phi}{n} + \frac{k2\pi}{n}) + i \sin(\frac{\phi}{n} + \frac{k2\pi}{n}))
\]

za $z = r(\cos\phi + i\sin\phi)$ in $k \in \{0, 1, \ldots, n-1\}$.

\section{Množice in preslikave}

\vprasanje{Kdaj sta množici $A$ in $B$ ekvipotentni?}

Kadar obstaja bijektivna preslikava $A \desno B$.

\vprasanje{Kdaj je množica števno neskončna in kdaj števna?}

Množica je števno neskončna, če je ekvipotentna z $\N$. Množica je števna, če je končna ali števno neskončna.

\section{Številska zaporedja}

\vprasanje{Kaj je zaporedje realnih števil?}

To je preslikava $\N \desno \R$.

\vprasanje{Kdaj zaporedje $(a_n)_n$ konvergira proti številu $a \in \R$? Kako v takem primeru označimo $a$?}

Kadar velja, da za vsak $\epsilon > 0$ obstaja $M \in \N$, da za vse $n \ge M$ velja $|a_n - a| < \epsilon$. Označimo $a = \lim_{n\desno \infty} a_n$.

\vprasanje{Kdaj je zaporedje konvergentno in kdaj divergentno?}

Zaporedje je konvergentno, če konvergira. Če ne kovergira, je divergentno.

\vprasanje{Kaj je $\epsilon$-ska okolica števila $a$? Kaj je okolica števila $a$?}

To je odprt interval $(a-\epsilon, a+\epsilon)$. Okolica je množica, v kateri je vsaj kakšna $\epsilon$-ska okolica $a$.

\vprasanje{Koliko limit ima lahko konvergentno zaporedje? Katera konvergentna zaporedja so omejena?}

Vsako konvergentno zaporedje ima natanko eno limito. Vsak konvergentna zaporedja so omejena.

\vprasanje{Kaj je stekališče zaporedja $(a_n)_n$?}

To je število $s \in \R$, za katero velja, da v vsaki okolici števila $s$ leži neskončno mnogo členov zaporedja.

\vprasanje{Kaj je naraščajoče in kaj monotono zaporedje?}

Zaporedje $(a_n)_n$ je naraščajoče, če velja $\forall n \in \N . a_{n+1} \ge a_n$.

Zaporedje je monotono, če je naraščajoče ali padajoče.

\vprasanje{Definiraj podzaporedje zaporedja $(a_n)_n$.}

Naj bo $(n_j)_j$ strogo naraščajoče zaporedje naravnih števil. Zaporedje $(a_{n_j})_j$ imenujemo podzaporedje zaporedja $(a_n)_n$.

\vprasanje{Kaj je \textit{rep zaporedja} $a_n$?}

To je vsako zaporedje $a_m, a_{m+1}, a_{m+2}, \ldots, m \in \N$.

\vprasanje{Povej izrek o sendviču.}

Naj bodo zaporedja $(a_n), (b_n), (c_n)$ takšna, da velja $a_n < b_n < c_n$ za vsak $n \in \N$ in denimo, da sta $(a_n)$ in $(c_n)$ konvergentni zaporedji z istima limitama. Potem je tudi $(b_n)$ konvergentno zaporedje z enako limito.

\vprasanje{Kaj je $\lim_{n\desno \infty} \sqrt[n]{n}$?}

1.

\vprasanje{Kaj je Cauchyjevo zaporedje?}

Naj bo $(a_n)$ zaporedje. Pravimo, da $a_n$ izpolnjuje Cauchyjev pogoj, če za vsak $\epsilon>0$ obstaja tak $M\in\N$, da za vse $n,m\ge M$ velja $|a_n-a_m| < \epsilon$. Če zaporedje izpolnjuje ta pogoj, je to Cauchyjevo zaporedje.

\vprasanje{Kdaj zaporedje konvergira proti neskončnosti?}

Kadar za vsak $K \in \R$ obstaja $M \in \N$, da za vsak $n\ge M$ velja $a_n > K$.

\vprasanje{Kakšni sta drugi imeni za zgornjo in spodnjo limito zaporedja? Kako sta definirani?}

Limes superior in limes inferior. To sta supremum in infimum množice vseh stekališč zaporedja.

\vprasanje{Ali tudi za kompleksna zaporedja velja, da je konvergentno natanko tedaj, ko izpolnjuje Cauchyjev pogoj?}

Da.

\section{Številske vrste}

\vprasanje{Kaj je številska vrsta? Kaj je splošni člen številske vrste?}

To je neskončna formalna vsota realnega zaporedja $(a_n)$: \[
	a_1 + a_2 + \ldots + a_k + \ldots = \sum_{k=1}^{\infty} a_k
\]

Splošni člen je število $a_k$.

\vprasanje{Kdaj številska vrsta $\vrsta$ konvergira?}

Kadar konvergira zaporedje delnih vsot $s_n = a_1 + a_2 + \ldots + a_n$.

\vprasanje{Kam konvergira geometrijska vrsta $\sum_{n=0}^{\infty} aq^n$?}

Če je $|q| < 1$, konvergira k $\frac{a}{1-q}$. Sicer ne konvergira.

\vprasanje{Povej Cauchyjev pogoj za vrsto $\vrsta$. Kakšna je limita zaporedja $(a_n)$, če vrsta $\vrsta$ konvergira?}

Vrsta konvergira natanko tedaj, kadar za vsak $\epsilon>0$ obstaja $n \in \N$, da za vse $n \ge N$ in za vse $p \in \N$ velja \[|a_{n+1} + a_{n+2} + \ldots + a_{n+p}| < \epsilon\].

V takem primeru je $\lim_{n\desno \infty} a_n = 0$.

\vprasanje{Kaj je harmonična vrsta?}

\[
	\sum_{n=1}^{\infty} \frac{1}{n}
\]
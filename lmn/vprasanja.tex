\section{Množice in preslikave}

\subsection{Osnovno o množicah}

\vprasanje{Kaj je \textit{ekstenzionalnost množic}?}

To je pravilo, ki trdi, da sta množici enaki, če imata iste elemente; oz. če je vsak element prve množice tudi element druge množice, in obratno.

\vprasanje{Kako zapišemo prazno množico in standardni enojec?}

Prazna množica: $\{\}$ ali $\emptyset$

Standardni enojec: $1 = \{()\}$

\vprasanje{Kaj je \textit{enojec}?}

Množica $A$ je enojec, kadar velja:
\begin{itemize}
	\item obstaja $x \in A$
	\item če $x \in A$ in $y \in A$, potem $x = y$
\end{itemize}

\subsection{Konstrukcije množic}

\vprasanje{Kaj sta zmnožek in vsota množic? Kako se drugače imenujeta?}

Zmnožek ali \textit{kartezični produkt} množic $A$ in $B$ je nova množica $A \x B$, katere elementi so urejeni pari $(x,y), x \in A, y \in B$.

Vsota ali \textit{koprodukt} množic $A$ in $B$ je množica $A + B$, katere elementi so $\inj{1} x$ za $x \in A$ ter $\inj{2} y$ za $y \in B$.

\vprasanje{Kaj je projekcija $\prj{i} u$ in kaj injekcija $\inj{i} x$?}

$\prj{i} u$ označuje $i$-to komponento urejenega para $u$.

$\inj{i} x$ je simbol, s katerim ločimo elemente različnih množic pri seštevanju.

\subsection{Preslikave}

\vprasanje{Katere tri komponente ima preslikava?}

Domeno, kodomeno in prirejanje.

Prirejanje mora biti: (za preslikavo $f: A \desno B$)
\begin{itemize}
	\item \textit{Celovito}: Za vsak $x \in A$ obstaja $y \in B$, ki mu je prirejen.
	\item \textit{Enolično}: Če sta elementu $x \in A$ prirejena elementa $y$ in $z \in B$, potem $y = z$.
\end{itemize}

\vprasanje{Kaj je eksponent množic $A$ in $B$?}

To je množica $B^A$, katere elementi so preslikave z domeno $A$ in kodomeno $B$.

\vprasanje{Koliko je preslikav $A \desno \emptyset$ in koliko preslikav $\emptyset \desno A$?}

Preslikava $\emptyset \desno A$ je ena sama.

Preslikav $A \desno \emptyset, A \ne \emptyset$ ni.

\subsection{Uporabne preslikave}

\vprasanje{Kaj je identiteta?}

Je preslikava $\id{A}: A \desno A$ za poljubno množico $A$; $\id{A}: x \slika x$.

\vprasanje{Kaj je kompozitum preslikav? Povej dve njegovi računski lastnosti.}

Kompozitum preslikav $f: A \desno B$ in $g: B \desno C$ je preslikava $g \circ f: A \desno C$, ki $x \slika g(f(x))$.

Kompozitum je asociativen in ima nevtralen element - identiteto.

\vprasanje{Kaj je \textit{evalvacija}? Kako jo še imenujemo?}

Evalvacija, \textit{aplikacija} ali \textit{uporaba} je preslikava, ki sprejme preslikavo in argument, ter preslikavo uporabi na argumentu.

$\text{ev}: B^A \x A \desno B$\\
$\text{ev}: (f, x) \slika f(x)$

\subsection{Izomorfizem}

\vprasanje{Kdaj je $f: A \desno B$ inverz $g: B \desno A$? Kako imenujemo preslikavo, ki ima inverz? Kako označimo inverz take preslikave?}

Kadar velja $f \circ g = \id{B}$ in $g \circ f = \id{A}$.

Preslikavo $f$ z inverzom imenujemo \textit{izomorfizem} in jo označimo z $f^{-1}$.

\vprasanje{Kaj velja za kompozitum dveh izomorfnih preslikav $f, g$? Kaj velja za njegov inverz?}

Tudi kompozitum je izomorfizem. Za njegov inverz velja:
$(g \circ f)^{-1} = f^{-1} \circ g^{-1}$

\vprasanje{Kdaj sta dve množici $A, B$ izomorfni? Kako to zapišemo?}

Množici sta izomorfni, če obstaja izomorfizem iz $A$ v $B$.

To zapišemo $A \cong B$.

\vprasanje{Katere tri lastnosti veljajo za $\cong$?}

Refleksivnost, simetričnost, tranzitivnost.

\section{Logika in simbolni zapis}

\vprasanje{Povej primer levo in primer desno asocirane operacije.}

Levo asocirana: $+, -, \x, \ldots$

Desno asocirana: $\potem, \desno, \ldots$

\vprasanje{Kaj so implicitni argumenti? Povej primer.}

To so argumenti operatorja, ki jih pogosto izpustimo; takrat mora bralec iz konteksta izvesti, kaj naj bi te argumenti bili; npr.\ množici v zapisu projekcije $\text{pr}_1^{A,B}$.

\vprasanje{Kakšna je razlika med implicitnimi argumenti in privzeto vrednostjo? Povej primer privzete vrednosti.}

Pri privzeti vrednosti smo dogovorjeni, s čim naj nadomestimo izpuščeno vrednost, pri implicitnih argumentih pa so vrednosti vedno drugačne in razvidne iz konteksta. Primer je $\log x = \log_{10} x$.

\subsection{Logične formule}

\vprasanje{Kateri dve vrsti logičnih formul poznamo? Kaj je razlika med njima?}

\begin{itemize}
	\item \textit{Izjavni račun}: Zapisujemo izjave z osnovnimi vezniki
	\item \textit{Predikatni račun}: Za zapisovanje uporabimo tudi predikate, t.j. operatorje $=, \ne, \le, \ldots, \forall, \exists$.
\end{itemize}

\vprasanje{Kako zapišemo resnico, neresnico, negacijo, konjunkcijo in disjunkcijo?}

\begin{itemize}
	\item Resnica: $\resnica$
	\item Neresnica: $\neresnica$
	\item Negacija: $\neg$
	\item Konjunkcija: $\land$
	\item Disjunkcija: $\lor$
\end{itemize}

\vprasanje{Kako se imenujeta argumenta pri implikaciji? Kako se jih na dolgo bere?}

\[
	\text{Antecedent} \potem \text{Konsekvent}
\]

Antecedent je zadosten pogoj za konsekvent.

Konsekvent je potreben pogoj za antecedent.

\vprasanje{Katera kvantifikatorja poznamo?}

Univerzalni kvantifikator ($\forall$) in eksistenčni kvantifikator ($\exists$).

\subsection{Definicije in dokazi}

\vprasanje{Kako dokažeš enolični obstoj? Kako ga označiš?}

Tako da dokažeš, da sta dve vrednosti, pri katerih pogoj drži, nujno enaki. Označimo z $\exists!$.

\vprasanje{Kaj je \textit{operator enoličnega opisa}?}

To je operator $\iota x \in A.\phi(x)$, s katerim določimo enolično opisan element; pod pogojem, da velja $\exists!x \in A. \phi(x)$.

\vprasanje{Kaj je kontekst? Kako vanj uvedemo novo spremenljivko?}

Kontekst je skupek vseh trenutno veljavnih simbolov in predpostavk. Vanj uvedemo novo spremenljivko z definicijo ($x := \ldots$) ali prosto ("`Naj bo $x \in A$"').

\vprasanje{Kako se s pravili vpeljave dokaže implikacijo, ekvivalenco, univerzalno izjavo in eksistenčno izjavo?}

Za implikacijo predpostavimo levo stran in dokažemo desno. Za ekvivalenco dokažemo obe implikaciji. Za univerzalno izjavo dokažemo $\phi(x)$ za poljuben $x \in A$. Za eksistenčno izjavo podamo nek $x$, dokažemo $x \in A$ in dokažemo $\phi(x)$.

\subsection{Boolova algebra}

\vprasanje{Kako imenujemo spremenljivko z vrednostjo iz $2 = \{\resnica, \neresnica\}$? Kakšna je Boolova preslikava? Kaj je tavtologija?}

Izjavna spremenljivka.

Boolova preslikava je vsaka preslikava $\underbrace{2 \x 2 \x 2 \x \ldots \x 2}_n \slika 2$.

Tavtologija je izjava, katere resničnostna vrednost je vedno resnica, ne glede na vrednost spremenljivk.

\vprasanje{Kakšna je disjunktna oblika izjave?}

\[
	\phi(p_1,p_2,\ldots,p_n) = c_1 \lor c_2 \lor \ldots \lor c_m
\]

Kjer so \[
	c_i = l_1 \land l_2 \land \ldots \land l_n
\]
in \[
	l_j = p_j \text{ ali } \lnot p_j
\]

\vprasanje{Kdaj je nabor logičnih veznikov poln? Povej primer polnega nabora.}

Kadar lahko s temi vezniki izrazimo vsako resničnostno tabelo; npr. $\land, \lor, \lnot$.

\vprasanje{Definiraj karakteristično preslikavo in potenčno množico.}

Karakteristična preslikava je preslikava $A \desno 2$.

Potenčna množica $P(A)$ je množica, katere elementi so podmnožice $A$.